\documentclass[10pt,twocolumn,letterpaper]{article}

\usepackage{cvpr}
\usepackage{times}
\usepackage{epsfig}
\usepackage{graphicx}
\usepackage{amsmath}
\usepackage{amssymb}
\usepackage{ctex}

%\usepackage{multicol}

\usepackage[top=1in, bottom=1in, left=1.25in, right=1.25in]{geometry}

\usepackage{lscape}

\usepackage{cite}

\usepackage[justification=centering]{caption}

\usepackage[breaklinks=true,bookmarks=false,colorlinks,
            linkcolor=red,
            anchorcolor=blue,
            citecolor=green,
            backref=page
            ]{hyperref}
            
\cvprfinalcopy
\def\cvprPaperID{****} 
\def\httilde{\mbox{\tt\raisebox{-.5ex}{\symbol{126}}}}
%\setcounter{page}{4321}
\bibliographystyle{plain}
% �����ϱ���ʾ�ο����ױ�ŵ�����
%\newcommand{\upcite}[1]{\textsuperscript{\textsuperscript{\cite{#1}}}}
\title{Simulation of specular surface imaging based
on computer graphics: application on a vision
inspection system}
\author{Shuang Sha\\\\ May 29,2018}

\begin{document}




\maketitle


\begin{abstract}
  The main purpose of this paper is to detect surface defects on reflecting industrial parts. A system based machine vision is designed to detect surface defects in geometry. Through specific lighting devices, defects can be imaged, and then image segmentation is processed. And a real-time inspection of reflective products is possible.
\end{abstract}

\section{Introduction}

Highly reflective surfaces inspection is a problem met frequently within the automatic control of industrial parts ~\cite{Sanderson1988}. Usually, this work require to be done manually by human. But the results of detection is impacted to human��s subjective and tiredness.

The human eye does not necessarily see all the defects. The error rate is relatively high.
The system described by this paper based machine vision is a particular lighting device system that enables efficient real time defect detection and some of the features design performed via computer graphics simulation~\cite{Seulin2002}.

\section{Specular surface imaging}
\subsection{Lighting principle}
Figure ~\ref{1} illustrates the lighting principle. In the first case, without defect, the surface reflects a dark zone of the lighting. In the second case, the defect deflects luminous rays coming from the luminous zone and so, the defect appears as a clear spot in a dark zone.

\begin{figure*}[!htb]
\centering
\includegraphics[width=0.8\textwidth]{1.jpg}
\caption{ Lighting principle.}
 \label{1}
\end{figure*}

\subsection{Implementation}
Through experiments, researchers can find that the size of the defect signature on the image depends on the distance between the light transition and the defect, as shown in Figure ~\ref{2}.

\begin{figure}[!htb]
\centering
\includegraphics[width=0.4\textwidth]{2.jpg}
\caption{Defect size variation.}
 \label{2}
\end{figure}


As shown in Figure ~\ref{3}, it represents a defect size (percentage of real size) versus the distance between two light transitions and the center of the defect (normalized by the defect dimensions) and the corresponding images.
\begin{figure}[!htb]
\centering
\includegraphics[width=0.4\textwidth]{3.jpg}
\caption{ Defect size variation versus distance to the first light transition (experimental results).}
 \label{3}
\end{figure}


\renewcommand\refname{Reference}
%\bibliographystyle{plain}
\bibliography{books}


\end{document}

