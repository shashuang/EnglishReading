\documentclass[10pt,twocolumn,letterpaper]{article}

\usepackage{cvpr}
\usepackage{times}
\usepackage{epsfig}
\usepackage{graphicx}
\usepackage{amsmath}
\usepackage{amssymb}

% Include other packages here, before hyperref.

% If you comment hyperref and then uncomment it, you should delete
% egpaper.aux before re-running latex.  (Or just hit 'q' on the first latex
% run, let it finish, and you should be clear).
\usepackage[breaklinks=true,bookmarks=false,colorlinks,
            linkcolor=red,
            anchorcolor=blue,
            citecolor=green,
            backref=page]{hyperref}

\cvprfinalcopy % *** Uncomment this line for the final submission

\def\cvprPaperID{****} % *** Enter the CVPR Paper ID here
\def\httilde{\mbox{\tt\raisebox{-.5ex}{\symbol{126}}}}

% Pages are numbered in submission mode, and unnumbered in camera-ready
%\ifcvprfinal\pagestyle{empty}\fi
%\setcounter{page}{4321}
\begin{document}

%%%%%%%%% TITLE
\title{Forecasting human dynamics from static images}

\author{Shuang Sha \\\\ June 31,2018}

\maketitle
%\thispagestyle{empty}

%%%%%%%%% ABSTRACT
\begin{abstract}
  This paper proposed that forecasting human dynamics from static images at the first time. Researchers proposed the 3D Pose Forecasting Network (3D-PFNet). They train our 3D-PFNet using a three-step training strategy to leverage a diverse source of training data, including image and video based human pose datasets and 3D motion capture (MoCap) data.
\end{abstract}

%%%%%%%%% BODY TEXT
\section{Introduction}

Human pose forecasting is the capability of predicting future human body dynamics from visual observations. For example, by looking at the left image of Figure ~\ref{fig:onecol}, we can predict that the next step of the tennis player, namely a forehand swing. As shown in the right image of Figure ~\ref{fig:onecol}, this is the sequence of upcoming pose.

\begin{figure}[!htb]
\begin{center}
   \includegraphics[width=0.8\linewidth]{1.jpg}
\end{center}
   \caption{Forecasting human dynamics from static images. Left:
the input image. Right: the sequence of upcoming poses.}
\label{fig:onecol}
\end{figure}

This paper presents the first study on human pose forecasting from static images~\cite{chao2017forecasting}. The chief task is to take a single RGB image and output a sequence of future human body poses. Firstly, as opposed to other forecasting tasks that assume a multi-frame input (e.g. videos) ~\cite{srivastava2015unsupervised}, this work assumes a single-frame input. Secondly, like most forecasting problems ~\cite{yuen2010data}, this work first represent the forecasted poses in the 2D image space, and then convert from 2D space to 3D space.

%------------------------------------------------------------------------
\section{Network architecture}

This paper proposed a deep recurrent network to predict human skeleton sequences, as shown in Figure~\ref{fig:twocol}. This network is divided into two components: (1) a 2D pose sequence generator that takes an input image and sequentially generate 2D body poses, where each pose is represented by heatmaps of keypoints; (2) a 3D skeleton converter that converts each 2D pose into a 3D skeleton.
\begin{figure}[!htb]
\begin{center}
   \includegraphics[width=0.8\linewidth]{2.jpg}
\end{center}
   \caption{A schematic view of the unrolled 3D-PFNet.}
\label{fig:twocol}
\end{figure}

{\small
\bibliographystyle{ieee}
\bibliography{books}
}

\end{document}
