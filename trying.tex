\documentclass{article}

\usepackage{ctex}

\usepackage{multicol}

\usepackage[top=1in, bottom=1in, left=1.25in, right=1.25in]{geometry}

\usepackage{lscape}

\author{Shuang Sha}

\date{April 21,2018}

\title{Tech Honcho Wants Innovation for the Bottom Billion\footnote{The source of the article is Scientific American}}




\begin{document}


\maketitle


Can we take advantage of technology innovation to improve poverty the standard of living of the poorest people? The answers of the author is definite. This paper tells us that we can use this magical power of invention and innovation to change the lives of people who really needed their lives changed, and also solve the problem of poverty or health care. But many people, oh yeah, we tried something and it failed. Because of part of the point here is that inventing anything is hard, we always failed in something. And people start giving up to trying, they think this way cannot solve the problem. We can try again and again, and there will always be a success, and it is that we want.



Many people were defeated again and again, and they selected to give up. As the saying goes, how do you see the rainbow without wind and rain. Failure is not terrible, it is terrible to give up easily. Difficulties are not problem, and you should think some ideas to solve difficulties. It probably is harder. But that doesn��t mean you shouldn��t try. And we should sum up experience from failure, and we can gain success in the end. Practice is the only criterion for testing truth. We only depend on our hard working, step by step, success can will beckon to us.




\end{document}

