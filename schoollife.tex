\documentclass{article}

\usepackage{ctex}

\usepackage{multicol}

\usepackage[top=1in, bottom=1in, left=1.25in, right=1.25in]{geometry}

\usepackage{lscape}

\author{Sha Shuang}

\date{April 11,2018}

\title{The School of life: Perfectionist Trap}


\begin{document}


\maketitle


The article I read is about the influence of school life on students' employment values. This article mainly describes that students in school life fall into the perfectionist trap, and that students build their ideals based on those who have achieved great success. The students only saw the glory of these successful persons, but ignored the hard work and sweat they had done behind them. They dreamed of how many accomplishments they could accomplish after graduation, so that they could lead a comfortable life. The reality, however, is not the case. Not only when looking for a job will encounter obstacles, and when the work will encounter various difficulties and problems. All of their expectations are broken. They are always worrying about their gains and losses, but they still don't work hard for them


Reading this article reminds me of my career experience. Before graduation, I also hope to find a good job, and life will not be too hard. However, the hard work of finding a job brought me back to reality. The difficulty of life makes me feel inferior, but I have not worked harder to make life better. Complaining about the hardships of life is obviously lack of ability and opportunity. Life can't see the glory on the surface, but also on the internal cost. I also understand and to success by sweat, keenly aware of the irrigation. Success is not effortless. It takes one minute on stage, the ability for ten years. We must strive to succeed without any difficulty. When opportunity comes, we can firmly grasp our own hands.
  


\end{document}

