\documentclass{article}

\usepackage{ctex}

\usepackage{multicol}

\usepackage[top=1in, bottom=1in, left=1.25in, right=1.25in]{geometry}

\usepackage{lscape}

\usepackage{graphicx}

\usepackage{subfigure}

\usepackage{cite}

\usepackage{color}

\bibliographystyle{plain}
% �����ϱ���ʾ�ο����ױ�ŵ�����
%\newcommand{\upcite}[1]{\textsuperscript{\textsuperscript{\cite{#1}}}}

\author{Shuang Sha}

\date{May 1,2018}

\title{China��s leading facial recognition beefs up nation��s surveillance network for security}




\begin{document}


\maketitle


Today, I read this article about the development of China��s facial recognition. Facial recognition is a new technology, its algorithms are mainly based on computer vision and deep learning, which uses big data to train the system.


There are many companies in China that are in the leading position in video surveillance or facial recognition. For example, Megvii, Hikvision and DaHua etc. Figure {\color{red}1} is photo courtesy of DaHua technology. China has been reportedly one of the world��s most popular destinations for investment in facial recognition. facial recognition is used in our daily life in China. There are many unmanned store in China that apply this technology to unlock doors, and we pay some money via We Chat or Alipay. In a high school, students can borrow some books using facial recognitions technology.




%����ͼƬ
\begin{figure}[htbp]
%\small
\centering
\includegraphics[width=0.5\textwidth]{FR.jpg}
\caption{Photo courtesy of Dahua Technology}
   \label{1}
\end{figure}

Following table {\color{red}1}, China has greater market value and faster growth rate than the United States in the same market.



%���Ʊ���
\begin{table}[htbp]
 \centering
  \caption{Facial recognition benefit comparison between China and US \cite{FacialRecognition}}
   \label{1}
   \par
\begin{tabular}{|c|c|c|}
 \hline
        & Value(billion($\$$)) & Rate($\%$) \\ \hline
  CHINA & 6.4 & 12.4 \\ \hline
  US    & 2.9 & 0.7  \\
  \hline
\end{tabular}
\end{table}

What��s more, facial recognition technology can help public security authorities to more precisely identify suspects via a real-time data pool. And polices can take advantage of this technology to capture criminal suspects. It will play a more active role in proactive security defense in the future.
Because of the different of each person, facial recognition could exit error. In the future, this technology can be more accurate to identify and not just his or her face.




\renewcommand\refname{Reference}
%\bibliographystyle{plain}
\bibliography{FacialRecognition}


\end{document}

