\documentclass{article}

\usepackage{ctex}

\usepackage{multicol}

\usepackage[top=1in, bottom=1in, left=1.25in, right=1.25in]{geometry}

\usepackage{lscape}

\usepackage{graphicx}

\usepackage{subfigure}

\usepackage{cite}

\usepackage{caption}

\usepackage{abstract}

\usepackage[colorlinks,
            linkcolor=red,
            anchorcolor=blue,
            citecolor=green
            ]{hyperref}
            
\usepackage{ragged2e}

\bibliographystyle{plain}
% �����ϱ���ʾ�ο����ױ�ŵ�����
%\newcommand{\upcite}[1]{\textsuperscript{\textsuperscript{\cite{#1}}}}

\author{Shuang Sha}

\date{May 21,2018}

\title{The experimental results of medication verification}




\begin{document}

\twocolumn
\maketitle

\begin{abstract}

\emph
{
This paper is dedicated to the development of an assistive computer vision-based system for medication verification before dispensing it to patients \cite{vmsystem2018}. The medication identification algorithm is based on the pill descriptor vectors with the reference ones, which are obtained by the attentive machine- learning algorithm during the training phase.}
\end{abstract}


\section{Experimental results}
As shown in Figure \ref{1}, this is a machine for medication verification. It consists of three components: pill scanning device, graphical interface and computer server\cite{vmsystem2018}. And medication verified need to put in a dispensing cup. Figure \ref{2} gives the results of the attention vision-based verification algorithm. In precision rate and recall rate of comparison between original and proposed,as shown in Table \ref{table1} we can find that proposed AV-based algorithm apparently improve in this two side. The proposed AV-based algorithm performed better due to the improved detection of feature-point areas and new descriptor set to form the PDVs.


%����ͼƬ
\begin{figure}[!htb]
%\small
\centering
\includegraphics[width=0.4\textwidth]{1.png}
\caption{Experimental system and images of pill dosages.}
   \label{1}
\end{figure}


%����ͼƬ
\begin{figure}[!htb]
%\small
\centering
\includegraphics[width=0.4\textwidth]{2.png}
\caption{Visualization results of the AV-based pill detection and verification.}
   \label{2}
\end{figure}

%�������
\begin{table}[!htb]
\caption{Accuracy of pill detection}
 \label{table1}
 \tabcolsep2.1pt
\renewcommand\arraystretch{1.3}
\begin{tabular}{@{\extracolsep{\fill}}|c|c|c|c|}
\hline
Method of medication verification     	          &Precision rate	    &Recall rate \\
\hline
Original AV-based algorithm\cite{Visual2018}      &0.91                 &0.88 \\
\hline
Proposed AV-based algorithm                       &0.93                 &0.95 \\
\hline
\end{tabular}
\end{table}

The research discovered that mantis 3D vision works differently from all previously known forms of biological 3D vision. Usually, humans and other animals see 3D in still images. But mantis are able to see 3D in moving things and find places where the picture is changing. Even if the scientists made the two eyes�� images completely different, mantises can still match up the places where things are changing\cite{prayingmantises2018}. And these things couldn't be done by humans. The new form of 3D vision is based on change over time instead of static images. At present, scientists begin to apply the principle of the 3D vision of praying mantises in low-power autonomous robots.

\renewcommand\refname{Reference}
%\bibliographystyle{plain}
\bibliography{books}


\end{document}

