\documentclass{article}

\usepackage{ctex}

\usepackage{multicol}

\usepackage[top=1in, bottom=1in, left=1.25in, right=1.25in]{geometry}

\usepackage{lscape}

\usepackage{graphicx}

\usepackage{subfigure}

\usepackage{cite}

\usepackage{color}

\bibliographystyle{plain}
% �����ϱ���ʾ�ο����ױ�ŵ�����
%\newcommand{\upcite}[1]{\textsuperscript{\textsuperscript{\cite{#1}}}}

\author{Shuang Sha}

\date{May 19,2018}

\title{3D vision discovered in praying mantis}




\begin{document}

\twocolumn
\maketitle

This paper tells a new discovery in the last research. Scientists discovered a new form of 3D vision in praying mantises, and applied in miniature glasses. As shown in Figure \ref{1}, this is a praying mantis. Human bless with stereo vision, but are not the only animals, including monkeys, cats, horses, owls and toads. These animals are mammal and birds. The praying mantis is the only insect we know.




%����ͼƬ
\begin{figure}[!htb]
%\small
\centering
\includegraphics[width=0.4\textwidth]{P.jpg}
\caption{Praying mantis (stock image).}
   \label{1}
\end{figure}

The research discovered that mantis 3D vision works differently from all previously known forms of biological 3D vision. Usually, humans and other animals see 3D in still images. But mantis are able to see 3D in moving things and find places where the picture is changing. Even if the scientists made the two eyes�� images completely different, mantises can still match up the places where things are changing\cite{prayingmantises2018}. And these things couldn't be done by humans. The new form of 3D vision is based on change over time instead of static images. At present, scientists begin to apply the principle of the 3D vision of praying mantises in low-power autonomous robots.

\renewcommand\refname{Reference}
%\bibliographystyle{plain}
\bibliography{books}


\end{document}

